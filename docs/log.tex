\documentclass{article}
\usepackage{amsmath}
\usepackage{enumitem}
\usepackage{hyperref}

\title{Project Log}
\author{Sean}
\date{\today}

\begin{document}

\maketitle

\section*{Introduction}
This document contains the log of activities and hours spent on the individual project for the course individualProject for Avans.

In this course the student needs to make a project that incorporates the graphical card to do some calculations. The student needs to use OpenGl or OpenCL to do this.

\section*{Goal}
The goal for this projcet is to use openCL to make a fluid simulation where water can be defined in a square space and the graphics card then calculats how the water flows.
The focus first will be to make this in 2 dimensions. If this is done and there is time left the project will be expanded to 3 dimensions.

\section*{Log of Activities}
\begin{enumerate}
    \item \textbf{Read brightspace and init project}
    \begin{itemize}
        \item \textbf{Activity} Read the brightspace course
        \item \textbf{Activity} Setup code environment
        \item \textbf{Activity} Setup git repository
        \item \textbf{Activity} Write template for this log
    \end{itemize}
    \item \textbf{Google what is OpenCL}
    \begin{itemize}
        \item \textbf{Activity} Read \href{https://www.khronos.org/api/index_2017/opencl/}{OpenCL landig page}
        \item \textbf{Activity} Read \href{https://medium.com/@bcrodrigues/what-is-opencl-14fbec353e09}{What is OpenCL}
    \end{itemize}
    \item \textbf{Google openCL tutorial}
    \begin{itemize}
        \item \textbf{Activity} Found \href{https://github.com/KhronosGroup/OpenCL-Guide/blob/main/chapters/getting_started_linux.md}{getting started linux} and followed part of tutorial with own knowledge to see if my envirement on new arch linux is working correctly.
        \item \textbf{Activity} Installed openCL headers
        \item \textbf{Activity} Followed \href{https://wiki.archlinux.org/title/GPGPU#OpenCL}{Arch wiki} to install openCL
        \item \textbf{Note} First program did not work. Had to google around and after rereading Arch wiki found I also had to install opencl-cover-mesa package isntead of only openc-nvidia.
        \item \textbf{Note} Added user to the video group using "sudo usermode -aG video sean", not sure if this was necessary
    \end{itemize}
    \item \textbf{Trying to use c++}
    \begin{itemize}
        \item \textbf{Activity} Found \href{https://github.com/KhronosGroup/OpenCL-CLHPP}{OpenCL-CLHPP} and looked at the example
        \item \textbf{Founding} Saw I have to add extra find package to my cmakelists (OpenCLHeaders, OpenCLICDLoader and OpenCLHeaderCpp)
        \item \textbf{Activity} After searching and trying things for about an hour an nothing working I remembered I can just add the raw hpp file to my project and use it that way.
    \end{itemize}
    \item \textbf{Search and follow basic tutorial of openCL}
    \begin{itemize}
        \item \textbf{Activity} Found and followed \href{https://programmerclick.com/article/47811146604/}{Simple start with OpenCL and C++}
        \item \textbf{Note} In the tutorial a device is selected. At first there was no know device. After searching and testing for about 45 minutes, the problem was that I did not restart my computer.....
    \end{itemize}
    \item \textbf{}
    \begin{itemize}
        \item \textbf{}
    \end{itemize}
\end{enumerate}

\section*{Hours Spent}
\begin{tabular}{|c|l|c|}
    \hline
    \# & Activity & Hours \\
    \hline
    1 & Read brightspace and init project & 1 \\
    2 & Google what is OpenCL & 0.5 \\
    3 & Google openCL tutorial & 1 \\
    4 & Trying to use c++ & 1.0 \\
    5 & Search and follow basic tutorial of openCL & 1.0 \\
    % Add more hours as needed
    \hline
    \multicolumn{2}{|r|}{\textbf{Total Hours}} & 4.5 \\
    \hline
\end{tabular}

\section*{Results}
\begin{itemize}
    \item Learned the basics of OpenCL.
    \item Understood how to use OpenCL for parallel computing.
    % Add more results as needed
\end{itemize}

\end{document}